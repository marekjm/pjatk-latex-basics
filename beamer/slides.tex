\documentclass[aspectratio=169]{beamer}

\usepackage[utf8]{inputenc}
\usepackage{textcomp}
\usepackage[official]{eurosym}
\usepackage[polish]{babel}
\usepackage{amsthm}
\usepackage{graphicx}
\usepackage[T1]{fontenc}
\usepackage{scrextend}
\usepackage{hyperref}
\usepackage{xcolor}
\usepackage{geometry}
\usepackage{listings}

\usetheme{-bjeldbak/beamerthemebjeldbak}

\title{Przykładowy tytuł}
\subtitle{Przykładowy podtytuł}
\author{Przemysław Rzykładowy}
\institute{Polsko-Japońska Akademia Technik Komputerowych}

\lstset{basicstyle=\ttfamily\color{black},
columns=fixed,
escapeinside={\%*}{*)},
inputencoding=utf8,
extendedchars=true,
moredelim=**[is][\color{red}]{@}{@}}

\begin{document}

{%
    \setbeamertemplate{headline}{}
    \frame{\titlepage}
}

\section{Wstęp}

\begin{frame}
    \frametitle{Przykładowy tytuł slajdu}
    \framesubtitle{Przykładowy podtytuł slajdu}

    Przykładowy tekst.
\end{frame}

\begin{frame}[fragile]
    \frametitle{Slajdy z kodem}
    \framesubtitle{Muszą być oznaczone przez \texttt{fragile}}

    \begin{lstlisting}
    std::cout << "Hello, World!\\n";
    \end{lstlisting}
\end{frame}

\begin{frame}
    \frametitle{Jeszcze jeden}

    Slajd.
\end{frame}

\section{Rozwinięcie}

\begin{frame}
    \frametitle{Jak na polskim w podstawówce}

    Też był wstęp, rozwinięcie i zakończenie.
\end{frame}

\section{Zakończenie}

\begin{frame}
    Końcowy slajd.
\end{frame}

\end{document}

\documentclass[11pt,twoside,a4paper,titlepage,onecolumn]{book}
% Można też zrobić "twocolumn", ale dziwnie to wygląda.

\usepackage[utf8]{inputenc}
\usepackage{textcomp}
\usepackage[official]{eurosym}
\usepackage[polish]{babel}
\usepackage{amsthm}
\usepackage{graphicx}
\usepackage[T1]{fontenc}
\usepackage{scrextend}
\usepackage{hyperref}
\usepackage{xcolor}
\usepackage[inline]{enumitem}
\usepackage{geometry}
\usepackage{rotating}
\usepackage{listings}
\usepackage{tabularx}
\usepackage{longtable}

\renewcommand*{\lstlistlistingname}{Spis listingów}

\author{Przemysław Rzykładowy\\John Doe\\Promotor: dr hab. X. Y. Zeciński, prof. PJATK}
\title{
    Przykładowy tytuł\\
    \large
    Przykładowy podtytuł
}

\lstset{basicstyle=\ttfamily\color{black},
columns=fixed,
escapeinside={\%*}{*)},
inputencoding=utf8,
extendedchars=true,
moredelim=**[is][\color{red}]{@}{@}}

\begin{document}

\begin{titlepage}                                                                                        
	\includegraphics[width=\textwidth]{pjwstk_logo}
	\begin{center}
		{\huge\texttt{Polsko-Japońska Akademia Technik Komputerowych}}
		{\huge\texttt{Zamiejscowy Wydział Informatyki w Gdańsku}}
	\end{center}
	\vspace{1cm}
	{\Large\textbf{Imię i nazwisko dyplomanta:} Przemysław Rzykładowy}
	\vspace{0.5cm}
	\newline
	{\Large\textbf{Nr albumu:} s12345}
	\vspace{0.5cm}
	\newline
	{\Large\textbf{Kierunek studiów:}
	Informatyka \hfill \textbf{Rodzaj studiów:} niestacjonarne}

	\vspace{1.5cm}
	\begin{center}
	{\huge\textsc{Praca dyplomowa}}
	\end{center}
	\vspace{1cm}

	~
	\newline
	{\Large
	\textbf{Temat pracy:} Przykładowy tytuł\\
	\phantom{\textbf{Temat pracy:}} Dalsza część przykładowego tytułu}
	\vspace{1cm}
	\newline
	{\Large\textbf{Temat w języku angielskim:} Example title\\
	\phantom{\textbf{Temat w języku angielskim:}} Example title extended}
	\vspace{1cm}
	\newline
	{\Large\textbf{Opiekun pracy:} dr hab. X. Y. Zeciński, prof. PJATK}
	\vspace{0.5cm}
	\newline
	{\Large\textbf{Wykonawcy:} Przemysław Rzykładowy i John Doe}
	\vspace{1.5cm}
	\newline
	{\Large\textbf{Streszczenie:} Bla bla bla.}
	\vspace*{\fill}
	\begin{center}
	Gdańsk, 2019 rok
	\end{center}
\end{titlepage}

\frontmatter
\tableofcontents
\listoffigures
\lstlistoflistings
\vspace*{\fill}
Praca została złożona za pomocą systemu \LaTeX.

\newpage

\mainmatter

\part{Pierwsza część}

\chapter{Pierwszy rozdział}

Jest taki język jak ,,Go''. Opisuje go książka \cite{Golang}. To jest potrzebne
tylko po to żeby pokazać jak działa komenda \texttt{cite}. A w ten sposób
\footnote{działają dopiski}.

\section{Pierwsza sekcja}

Lorem ipsum dolor sit amet, consectetur adipiscing elit. Phasellus eget ornare
erat, ut semper felis. Donec varius eu lacus tempor eleifend. Quisque
condimentum lobortis velit malesuada fringilla. Pellentesque eu justo tellus.
Maecenas pharetra urna nulla, quis suscipit nunc tristique id. Vivamus
scelerisque at ex vestibulum condimentum. Maecenas a arcu finibus, viverra elit
eu, efficitur dui. Ut condimentum massa aliquam tellus tincidunt, ut mollis
libero tempus. Sed semper, turpis ut sagittis porttitor, augue turpis venenatis
ex, vitae volutpat ipsum eros sed libero. In elementum tortor at neque dictum,
eu interdum nisi consequat. Cras convallis, eros quis malesuada sodales, justo
justo cursus tellus, ac ultrices tellus ex at nunc. Maecenas nunc ipsum, auctor
in augue vitae, tincidunt vehicula lorem. Nam id tincidunt dolor. Aliquam
aliquam vitae lorem nec fermentum. Mauris bibendum faucibus velit, sed tincidunt
est facilisis sed.

Etiam semper gravida diam placerat rhoncus. Pellentesque in eros sit amet metus
tincidunt laoreet. Aenean consectetur in purus vel dictum. Curabitur a nibh
augue. Nullam mauris ante, egestas quis molestie vitae, molestie vitae lorem.
Quisque sit amet metus congue, elementum mi eget, cursus magna. Nam tempus massa
id bibendum bibendum. Vivamus ac erat rhoncus, volutpat mi ac, sodales enim.
Morbi aliquet malesuada dolor, non convallis odio porta quis. Nulla facilisi.
Nunc vitae sodales tortor, id mattis erat. Aliquam vestibulum, odio in gravida
dapibus, purus leo sodales felis, a sollicitudin tortor nulla eget nisl. Nam
aliquam, mauris eu varius faucibus, libero arcu rhoncus erat, sed scelerisque
velit felis ut nulla. In erat justo, sagittis nec aliquam vel, blandit vel
ante. Suspendisse maximus, sem in vulputate molestie, dolor urna rhoncus
mauris, euismod efficitur metus nisi non purus. Nullam et ultricies augue, non
sagittis turpis.

\section{Druga sekcja}

Lorem ipsum dolor sit amet, consectetur adipiscing elit. Phasellus eget ornare
erat, ut semper felis. Donec varius eu lacus tempor eleifend. Quisque
condimentum lobortis velit malesuada fringilla. Pellentesque eu justo tellus.
Maecenas pharetra urna nulla, quis suscipit nunc tristique id. Vivamus
scelerisque at ex vestibulum condimentum. Maecenas a arcu finibus, viverra elit
eu, efficitur dui. Ut condimentum massa aliquam tellus tincidunt, ut mollis
libero tempus. Sed semper, turpis ut sagittis porttitor, augue turpis venenatis
ex, vitae volutpat ipsum eros sed libero. In elementum tortor at neque dictum,
eu interdum nisi consequat. Cras convallis, eros quis malesuada sodales, justo
justo cursus tellus, ac ultrices tellus ex at nunc. Maecenas nunc ipsum, auctor
in augue vitae, tincidunt vehicula lorem. Nam id tincidunt dolor. Aliquam
aliquam vitae lorem nec fermentum. Mauris bibendum faucibus velit, sed tincidunt
est facilisis sed.

Etiam semper gravida diam placerat rhoncus. Pellentesque in eros sit amet metus
tincidunt laoreet. Aenean consectetur in purus vel dictum. Curabitur a nibh
augue. Nullam mauris ante, egestas quis molestie vitae, molestie vitae lorem.
Quisque sit amet metus congue, elementum mi eget, cursus magna. Nam tempus massa
id bibendum bibendum. Vivamus ac erat rhoncus, volutpat mi ac, sodales enim.
Morbi aliquet malesuada dolor, non convallis odio porta quis. Nulla facilisi.
Nunc vitae sodales tortor, id mattis erat. Aliquam vestibulum, odio in gravida
dapibus, purus leo sodales felis, a sollicitudin tortor nulla eget nisl. Nam
aliquam, mauris eu varius faucibus, libero arcu rhoncus erat, sed scelerisque
velit felis ut nulla. In erat justo, sagittis nec aliquam vel, blandit vel
ante. Suspendisse maximus, sem in vulputate molestie, dolor urna rhoncus
mauris, euismod efficitur metus nisi non purus. Nullam et ultricies augue, non
sagittis turpis.

\subsection{Pierwsza podsekcja}

Lorem ipsum dolor sit amet, consectetur adipiscing elit. Phasellus eget ornare
erat, ut semper felis. Donec varius eu lacus tempor eleifend. Quisque
condimentum lobortis velit malesuada fringilla. Pellentesque eu justo tellus.
Maecenas pharetra urna nulla, quis suscipit nunc tristique id. Vivamus
scelerisque at ex vestibulum condimentum. Maecenas a arcu finibus, viverra elit
eu, efficitur dui. Ut condimentum massa aliquam tellus tincidunt, ut mollis
libero tempus. Sed semper, turpis ut sagittis porttitor, augue turpis venenatis
ex, vitae volutpat ipsum eros sed libero. In elementum tortor at neque dictum,
eu interdum nisi consequat. Cras convallis, eros quis malesuada sodales, justo
justo cursus tellus, ac ultrices tellus ex at nunc. Maecenas nunc ipsum, auctor
in augue vitae, tincidunt vehicula lorem. Nam id tincidunt dolor. Aliquam
aliquam vitae lorem nec fermentum. Mauris bibendum faucibus velit, sed tincidunt
est facilisis sed.

Etiam semper gravida diam placerat rhoncus. Pellentesque in eros sit amet metus
tincidunt laoreet. Aenean consectetur in purus vel dictum. Curabitur a nibh
augue. Nullam mauris ante, egestas quis molestie vitae, molestie vitae lorem.
Quisque sit amet metus congue, elementum mi eget, cursus magna. Nam tempus massa
id bibendum bibendum. Vivamus ac erat rhoncus, volutpat mi ac, sodales enim.
Morbi aliquet malesuada dolor, non convallis odio porta quis. Nulla facilisi.
Nunc vitae sodales tortor, id mattis erat. Aliquam vestibulum, odio in gravida
dapibus, purus leo sodales felis, a sollicitudin tortor nulla eget nisl. Nam
aliquam, mauris eu varius faucibus, libero arcu rhoncus erat, sed scelerisque
velit felis ut nulla. In erat justo, sagittis nec aliquam vel, blandit vel
ante. Suspendisse maximus, sem in vulputate molestie, dolor urna rhoncus
mauris, euismod efficitur metus nisi non purus. Nullam et ultricies augue, non
sagittis turpis.

\subsubsection{Pierwsza podpodsekcja}

Lorem ipsum dolor sit amet, consectetur adipiscing elit. Phasellus eget ornare
erat, ut semper felis. Donec varius eu lacus tempor eleifend. Quisque
condimentum lobortis velit malesuada fringilla. Pellentesque eu justo tellus.
Maecenas pharetra urna nulla, quis suscipit nunc tristique id. Vivamus
scelerisque at ex vestibulum condimentum. Maecenas a arcu finibus, viverra elit
eu, efficitur dui. Ut condimentum massa aliquam tellus tincidunt, ut mollis
libero tempus. Sed semper, turpis ut sagittis porttitor, augue turpis venenatis
ex, vitae volutpat ipsum eros sed libero. In elementum tortor at neque dictum,
eu interdum nisi consequat. Cras convallis, eros quis malesuada sodales, justo
justo cursus tellus, ac ultrices tellus ex at nunc. Maecenas nunc ipsum, auctor
in augue vitae, tincidunt vehicula lorem. Nam id tincidunt dolor. Aliquam
aliquam vitae lorem nec fermentum. Mauris bibendum faucibus velit, sed tincidunt
est facilisis sed.

Etiam semper gravida diam placerat rhoncus. Pellentesque in eros sit amet metus
tincidunt laoreet. Aenean consectetur in purus vel dictum. Curabitur a nibh
augue. Nullam mauris ante, egestas quis molestie vitae, molestie vitae lorem.
Quisque sit amet metus congue, elementum mi eget, cursus magna. Nam tempus massa
id bibendum bibendum. Vivamus ac erat rhoncus, volutpat mi ac, sodales enim.
Morbi aliquet malesuada dolor, non convallis odio porta quis. Nulla facilisi.
Nunc vitae sodales tortor, id mattis erat. Aliquam vestibulum, odio in gravida
dapibus, purus leo sodales felis, a sollicitudin tortor nulla eget nisl. Nam
aliquam, mauris eu varius faucibus, libero arcu rhoncus erat, sed scelerisque
velit felis ut nulla. In erat justo, sagittis nec aliquam vel, blandit vel
ante. Suspendisse maximus, sem in vulputate molestie, dolor urna rhoncus
mauris, euismod efficitur metus nisi non purus. Nullam et ultricies augue, non
sagittis turpis.


\part{Część druga}

\chapter{Jeszcze jeden rozdział}

W części drugiej też coś jest.

Lorem ipsum dolor sit amet, consectetur adipiscing elit. Phasellus eget ornare
erat, ut semper felis. Donec varius eu lacus tempor eleifend. Quisque
condimentum lobortis velit malesuada fringilla. Pellentesque eu justo tellus.
Maecenas pharetra urna nulla, quis suscipit nunc tristique id. Vivamus
scelerisque at ex vestibulum condimentum. Maecenas a arcu finibus, viverra elit
eu, efficitur dui. Ut condimentum massa aliquam tellus tincidunt, ut mollis
libero tempus. Sed semper, turpis ut sagittis porttitor, augue turpis venenatis
ex, vitae volutpat ipsum eros sed libero. In elementum tortor at neque dictum,
eu interdum nisi consequat. Cras convallis, eros quis malesuada sodales, justo
justo cursus tellus, ac ultrices tellus ex at nunc. Maecenas nunc ipsum, auctor
in augue vitae, tincidunt vehicula lorem. Nam id tincidunt dolor. Aliquam
aliquam vitae lorem nec fermentum. Mauris bibendum faucibus velit, sed tincidunt
est facilisis sed.

\begin{figure}
    \centering
    \includegraphics[width=14cm]{pjwstk_logo}
    \caption{Logo uczelni}
    \label{logo_uczelni}
\end{figure}

Etiam semper gravida diam placerat rhoncus. Pellentesque in eros sit amet metus
tincidunt laoreet. Aenean consectetur in purus vel dictum. Curabitur a nibh
augue. Nullam mauris ante, egestas quis molestie vitae, molestie vitae lorem.
Quisque sit amet metus congue, elementum mi eget, cursus magna. Nam tempus massa
id bibendum bibendum. Vivamus ac erat rhoncus, volutpat mi ac, sodales enim.
Morbi aliquet malesuada dolor, non convallis odio porta quis. Nulla facilisi.
Nunc vitae sodales tortor, id mattis erat. Aliquam vestibulum, odio in gravida
dapibus, purus leo sodales felis, a sollicitudin tortor nulla eget nisl. Nam
aliquam, mauris eu varius faucibus, libero arcu rhoncus erat, sed scelerisque
velit felis ut nulla. In erat justo, sagittis nec aliquam vel, blandit vel
ante. Suspendisse maximus, sem in vulputate molestie, dolor urna rhoncus
mauris, euismod efficitur metus nisi non purus. Nullam et ultricies augue, non
sagittis turpis.

Etiam semper gravida diam placerat rhoncus. Pellentesque in eros sit amet metus
tincidunt laoreet. Aenean consectetur in purus vel dictum. Curabitur a nibh
augue. Nullam mauris ante, egestas quis molestie vitae, molestie vitae lorem.
Quisque sit amet metus congue, elementum mi eget, cursus magna. Nam tempus massa
id bibendum bibendum. Vivamus ac erat rhoncus, volutpat mi ac, sodales enim.
Morbi aliquet malesuada dolor, non convallis odio porta quis. Nulla facilisi.
Nunc vitae sodales tortor, id mattis erat. Aliquam vestibulum, odio in gravida
dapibus, purus leo sodales felis, a sollicitudin tortor nulla eget nisl. Nam
aliquam, mauris eu varius faucibus, libero arcu rhoncus erat, sed scelerisque
velit felis ut nulla. In erat justo, sagittis nec aliquam vel, blandit vel
ante. Suspendisse maximus, sem in vulputate molestie, dolor urna rhoncus
mauris, euismod efficitur metus nisi non purus. Nullam et ultricies augue, non
sagittis turpis.


\chapter{Kolejny rozdział}

Najlepiej jeśli to coś podzielić na rozdziały...

Lorem ipsum dolor sit amet, consectetur adipiscing elit. Phasellus eget ornare
erat, ut semper felis. Donec varius eu lacus tempor eleifend. Quisque
condimentum lobortis velit malesuada fringilla. Pellentesque eu justo tellus.
Maecenas pharetra urna nulla, quis suscipit nunc tristique id. Vivamus
scelerisque at ex vestibulum condimentum. Maecenas a arcu finibus, viverra elit
eu, efficitur dui. Ut condimentum massa aliquam tellus tincidunt, ut mollis
libero tempus. Sed semper, turpis ut sagittis porttitor, augue turpis venenatis
ex, vitae volutpat ipsum eros sed libero. In elementum tortor at neque dictum,
eu interdum nisi consequat. Cras convallis, eros quis malesuada sodales, justo
justo cursus tellus, ac ultrices tellus ex at nunc. Maecenas nunc ipsum, auctor
in augue vitae, tincidunt vehicula lorem. Nam id tincidunt dolor. Aliquam
aliquam vitae lorem nec fermentum. Mauris bibendum faucibus velit, sed tincidunt
est facilisis sed.

\section{Znowu sekcja}

...i żeby te rozdziały miały sekcje.

Lorem ipsum dolor sit amet, consectetur adipiscing elit. Phasellus eget ornare
erat, ut semper felis. Donec varius eu lacus tempor eleifend. Quisque
condimentum lobortis velit malesuada fringilla. Pellentesque eu justo tellus.
Maecenas pharetra urna nulla, quis suscipit nunc tristique id. Vivamus
scelerisque at ex vestibulum condimentum. Maecenas a arcu finibus, viverra elit
eu, efficitur dui. Ut condimentum massa aliquam tellus tincidunt, ut mollis
libero tempus. Sed semper, turpis ut sagittis porttitor, augue turpis venenatis
ex, vitae volutpat ipsum eros sed libero. In elementum tortor at neque dictum,
eu interdum nisi consequat. Cras convallis, eros quis malesuada sodales, justo
justo cursus tellus, ac ultrices tellus ex at nunc. Maecenas nunc ipsum, auctor
in augue vitae, tincidunt vehicula lorem. Nam id tincidunt dolor. Aliquam
aliquam vitae lorem nec fermentum. Mauris bibendum faucibus velit, sed tincidunt
est facilisis sed.


\bibliographystyle{ieeetr}
\bibliography{biblio}

\part{Załączniki}

\appendix

\chapter{Załącznik pierwszy}

W dodatkach można normalnie stosować rozdziały, sekcje, itd.

\chapter{Załącznik drugi}

Tutaj jakoś przykładowy kod:

\begin{lstlisting}[caption={Przykładowy listing},captionpos=b]
	std::cout << "Hello, World!\n";
\end{lstlisting}

\end{document}
